\HeaderA{array.yourcast}{Array generation tool for YourCast}{array.yourcast}
%
\begin{Description}\relax
Creates array from YourCast output for each
geographical unit.
\end{Description}
%
\begin{Usage}
\begin{verbatim}
array.yourcast(x, unlog=FALSE)
\end{verbatim}
\end{Usage}
%
\begin{Arguments}
\begin{ldescription}

\item[\code{x}] A \code{\LinkA{yourcast}{yourcast}} output object.

\item[\code{unlog}] Logical. Should the dependent variable be unlogged? Default: \code{FALSE}.
\end{ldescription}
\end{Arguments}
%
\begin{Value}
Creates array(s) from \code{\LinkA{yourcast}{yourcast}} output. One array is created per geographic area. If there is one geographic area, an array is returned; else, a named list of arrays is returned. Each array is of size \eqn{T}{} (number of times) by \eqn{A}{} (number of ages) by 3, where the last dimension captures the type of data: \code{y} (observed values), \code{yhat} (predicted values), and \code{comb} (observed values in-sample and predicted values out-of-sample).
\end{Value}
%
\begin{Author}\relax
Konstantin Kashin \email{kkashin@fas.harvard.edu}.
\end{Author}
%
\begin{References}\relax
\url{http://gking.harvard.edu/yourcast}
\end{References}
%
\begin{SeeAlso}\relax
\code{\LinkA{yourcast}{yourcast}} function and documentation
(\code{help(yourcast)})
\end{SeeAlso}
%
\begin{Examples}
\begin{ExampleCode}
# Run Lee-Carter model for Figure 2.6 in Demographic Forecasting
data(chp.2.6.1)
ff.allc <- log(allc2/popu2) ~  time	
ylc.allc <- yourcast(formula=ff.allc, dataobj=chp.2.6.1, model="LC",
                       elim.collinear=FALSE,
                       sample.frame=c(1950,2000,2001,2060))
                       
yc.array <- array.yourcast(ylc.allc)
dimnames(yc.array)

# predicted mortality rates (observed in-sample and predicted out-of-sample)
yc.array[,,"comb"]
\end{ExampleCode}
\end{Examples}

\HeaderA{plot.yourcast}{Plot generation tool for YourCast}{plot.yourcast}
%
\begin{Description}\relax
Creates graphics from YourCast output for each
geographical unit and prints to the device window or a .pdf file in
the specified directory.
\end{Description}
%
\begin{Usage}
\begin{verbatim}
## S3 method for class 'yourcast'
plot(x, plots=c("age","time"), title=NULL, subtitle=NULL, 
     age.opts=list(), time.opts=list(), threedim.opts=list(),totcount.opts=list(), 
     ex0.opts=list(),print="device", print.args=list(), dv.log=NA, ...)
\end{verbatim}
\end{Usage}
%
\begin{Arguments}
\begin{ldescription}

\item[\code{x}] A \code{\LinkA{yourcast}{yourcast}} output object.

\item[\code{plots}] Vector of strings specifying the plot types to be generated. The available plots are age profile (\code{"age"}), time profile (\code{"time"}), 3D age/time profile (\code{"threedim"}), total counts by year (\code{"totcounts"}), and life expectancy at birth (\code{"ex0"}). The order in which the plots are specified is the order in which they are generated in the device. A maximum of 3 plot types is allowed. Default: \code{c("age","time")}.

\item[\code{title}] String. Main title for the plots. Concatenated with country name if \code{dataobj\$G.names} exists. Default: \code{NULL}.

\item[\code{subtitle}] String. Subtitle for the plots. Default: \code{NULL}.

\item[\code{age.opts}] A list of options for the age profile plot. See `Details'.

\item[\code{time.opts}] A list of options for the time profile plot. See `Details'.

\item[\code{threedim.opts}] A list of options for the 3D plot. See `Details'.

\item[\code{totcount.opts}] A list of options for the total counts plot. See `Details'.

\item[\code{ex0.opts}] A list of options for the life expectancy at birth plot. See `Details'.

\item[\code{print}] String specifying whether graphical output should be
displayed sequentially on a device window (\code{"device"}) or saved
directly to a \file{.pdf} file in the \code{dpath} (\code{"pdf"}). 
Default: \code{"device"}.

\item[\code{print.args}] A list of options for the device. See `Details'.

\item[\code{dv.log}] Logical. Is the dependent variable logged? If \code{NA}, the function parses the formula in the \code{\LinkA{yourcast}{yourcast}} object to determine if the dependent variable is logged. Default: \code{NA}.

\item[\code{...}] Further arguments passed to or from other methods.

\end{ldescription}
\end{Arguments}
%
\begin{Details}\relax
Plots age profiles, time profiles, 3D age/time profiles, total counts (by time), and life expectancy at birth from \code{\LinkA{yourcast}{yourcast}} output. Total count (by time) plots should only be used if forecasting counts (eg. population or death counts) and life expectancy at birth should only be used if forecasting mortality rates - both are common quantities of interest calculated from these forecasts and are thus included in the function. See \code{\LinkA{lifetable}{lifetable}} for details of life expectancy at birth calculations from mortality rates. All plots are created using the \code{\LinkA{ggplot2}{ggplot2}} package, except for the 3D age/time profile, which is created using the \code{wireframe} function from the \code{lattice} library.

The function supports multiple plots in each call (up to 3) and arranges them horizontally in the order specified in \code{plots} argument. The device window is automatically resized to accomodate different number of plots, albeit device width and height may be specified manually through the \code{print.args} option (see below).

Plots are titled with the \code{title} and the \code{G.names} dataframe if it was supplied to \code{\LinkA{yourcast}{yourcast}} in the \code{dataobj}. For example if \code{title="Respiratory Infections"} and the geographic identifier for that region is matched with \code{"Belize"}, the plot will be titled ``Respiratory Infections, Belize''. One or both labels will be utilized by the function if available. A subtitle may also be specified through \code{subtitle}.

It is important to note that \code{plot.yourcast} will only work if
all cross sections within the same geographic unit are of the same
dimensions. If, for example, a cross section for one age group has
fewer yearly observations than another from the same group, these
missing years must be filled in with \code{NA}, even if they occur in
the beginning of the sample period. This does not hold across
geographic units, however.

Finally, note that \code{plot.yourcast} opens a new device window for each new
plot. This is done so that the size of the device can be controlled to
keep the side-by-side plots from appearing distorted when launched.

\strong{Options for `age', `time', `totcount', and `ex0' plots}

Options for the age profile, time profile, total count, and life expectancy at birth plots may be specified using the \code{age.opts}, \code{time.opts}, \code{totcount.opts}, and \code{ex0.opts} arguments, respectively, which are lists with any of following components:

\begin{description}

\item[\code{xlab}] String specifying the x-axis label. Default is \code{"Age"} for `age' plots and \code{"Time"} for `time', `totcount', and `ex0' plots.

\item[\code{ylab}] String specifying the y-axis label. Default is \code{"Data and
    Forecasts"} for `age' plots and \code{"Forecasts"} for `time' plots.

\item[\code{insamp.obs}] Logical. Should the observed data be plotted? Default for `age', `totcounts', `ex0' is \code{TRUE}; default for  `time' is \code{FALSE}.

\item[\code{insamp.predict}] Logical. Should the predicted data be plotted for the in-sample period? Default: \code{TRUE}.

\end{description}


Additionally, the following options may be passed to \code{age.opts} and \code{time.opts} lists:
\begin{description}

\item[\code{age.select}] A numeric vector listing the ages to be plotted. If a scalar is supplied, it is used as the step size in a sequence of ages between the minimum and the maximum age. If \code{NULL}, all ages are plotted. Default: \code{NULL}.

\item[\code{time.select}] A numeric vector listing the times to be plotted. If a scalar is supplied, it is used as the step size in a sequence of times between the minimum and the maximum time. If \code{NULL}, all times are plotted. Default: \code{NULL}.

\item[\code{unlog}] Logical. Should the dependent variable be unlogged for `age' and `time' plots? Ignored if \code{dv.log} is set to \code{FALSE}. Default: \code{FALSE}.

\end{description}
 

Option \code{age.select} may be included in the \code{totcount.opts} list, but it functions differently than for the `age' and `time' plots. Instead of designating the ages to plot, \code{age.select} should be a numeric vector listing the ages to be summed over in the calculation of total counts over time. 

For `ex0' plots, options \code{ax}, \code{a0}, and \code{nx} that control the calculation of life tables from mortality rates in \code{\LinkA{yourcast}{yourcast}} objects may be included in the \code{ex0.opts} list. By default, \code{ax} is set to 0.5, \code{a0} is set to 0.07, and \code{nx} is set to \code{NULL}. See \code{\LinkA{lifetable}{lifetable}} for additional information.

\strong{Options for `threedim' plot}

Options for the 3D age/time profile may be specified through \code{threedim.opts}. The argument must be a list with any of the following components:
\begin{description}

\item[\code{xlab}] String specifying the x-axis label. This corresponds to the time dimension. Default: \code{"Time"}.

\item[\code{ylab}] String specifying the y-axis label. This corresponds to the age dimension. Default: \code{"Age"}.

\item[\code{zlab}] String specifying the z-axis label. This corresponds to the dependent variable. Default: \code{"Forecasts"}

\item[\code{insamp.pred}] Logical. If \code{TRUE}, the forecasted data is plotted for the in-sample period. If \code{FALSE}, the observed data is plotted instead. Default: \code{TRUE}.

\item[\code{unlog}] Should the dependent variable be unlogged? Ignored if \code{dv.log} is set to \code{FALSE}. Default: \code{FALSE}.

\item[\code{args.wireframe}] A list of arguments (must be labeled) to be
passed to \code{wireframe}. Note that some arguments to \code{wireframe}
such as \code{zlab} and \code{screen} should be made as arguments to
\code{threedim.opts}; they will be overwritten if found in
\code{args.wireframe}.

\item[\code{screen}] List with three elements `x', `y',
and `z' that rotate the viewing angle for the 3D plot. Default: \code{list(z=-40, x=-60, y=0)}.

\end{description}
 


\strong{Print options}

The function prints sequentially to the device or saves \file{.pdf} files
with the requested plot for each geographic unit in the sample. If
requested, \file{.pdf} files will be saved in a specified directory or
the working directory. Use \code{print.args} to set print options:
\begin{description}

\item[\code{height}] Height of device. Default is 5 inches.

\item[\code{width}] Width of device. Default width adjusts automatically according to number of plots.

\item[\code{filename}] A string or vector of strings denoting the filename(s) of \file{.pdf}s created for each geographical area if \code{print="pdf"}. If a string, the \code{csid} code is automatically appended if multiple geographies so as to ensure unique filenames. If the vector is of the length equal to the number of geographies, the \code{csid} is not appended, albeit the order of the labels should be the same as the order of the areas in \code{dataobj\$data}. Default: \code{NULL}, in which case plots are titled as \code{plotgggg.pdf}, where \code{gggg} denotes the \code{csid}.

\item[\code{dpath}] A string giving the directory where \file{.pdf} files will be output if \code{print="pdf"}. Default: \code{getwd()}.

\end{description}
 
\end{Details}
%
\begin{Value}
Device windows with requested plots or \file{.pdf} files saved
in the \code{dpath}.
\end{Value}
%
\begin{Author}\relax
Konstantin Kashin \email{kkashin@fas.harvard.edu}, based on an earlier version by Jon Bischof.
\end{Author}
%
\begin{References}\relax
\url{http://gking.harvard.edu/yourcast}
\end{References}
%
\begin{SeeAlso}\relax
\code{\LinkA{yourcast}{yourcast}} function and documentation
(\code{help(yourcast)})
\end{SeeAlso}
%
\begin{Examples}
\begin{ExampleCode}
# Run Lee-Carter model for Figure 2.6 in Demographic Forecasting
data(chp.2.6.1)
ff.allc <- log(allc2/popu2) ~  time	
ylc.allc <- yourcast(formula=ff.allc, dataobj=chp.2.6.1, model="LC",
                       elim.collinear=FALSE,
                       sample.frame=c(1950,2000,2001,2060))
# default
plot(ylc.allc,title="All causes of death")

## Not run: 
# output plots to pdf
plot(ylc.allc,title="All causes of death", print="pdf")

# output plots to pdf with names "acd"+csid
plot(ylc.allc,title="All causes of death", print="pdf",
print.args=list(filename="acd"))

# order matters
plot(ylc.allc,plots=c("time","age"),title="All causes of death")
plot(ylc.allc,plots=c("age","time"),title="All causes of death")
## End(Not run)

# remove in-sampled predicted data from age profile plot
plot(ylc.allc,title="All causes of death",age.opts=list(insamp.predict=FALSE))

# plot every 10th age instead of every 5th age for age profile plot
plot(ylc.allc,title="All causes of death",
age.opts=list(insamp.predict=FALSE,age.select=10))

## Not run: 
# plot every 5th year instead of every year for time series plot
plot(ylc.allc,title="All causes of death",
age.opts=list(insamp.predict=FALSE),time.opts=list(time.select=5))

# only plot data from 2000+
plot(ylc.allc,title="All causes of death",
age.opts=list(insamp.predict=FALSE, time.select=seq(2001,2060,5)),
time.opts=list(time.select=seq(2001,2060,5)))

# plot unlogged mortality rates
plot(ylc.allc,title="All causes of death",
age.opts=list(insamp.predict=FALSE, unlog=TRUE),time.opts=list(unlog=TRUE))

# plot `threedim' plot in addition to age and time profiles
plot(ylc.allc,plots=c("threedim","age","time"),
title="All causes of death",age.opts=list(insamp.predict=FALSE))

# plot life expectancy at birth in addition to age and time profiles
# since these are 5-year age groups, assume ax=2.5 and a0=1.907
plot(ylc.allc,plots=c("ex0","age","time"),title="All causes of death",
age.opts=list(insamp.predict=FALSE), ex0.opts=list(a0=1.907,ax=2.5))
## End(Not run)

\end{ExampleCode}
\end{Examples}
